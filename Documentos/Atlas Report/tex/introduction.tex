\chapter*{Introduction}

This document will be used to keep track of the Atlas project, a Feature Modeling tool with two primary purposes that both drive its design and set it apart from the existing technology. These purposes are a) to be used in an educational setting as a tool for teaching basic Software Product Line (SPL) concepts and b) to be used as a distributed research tool in maintaining a repository of publicly-accessible feature models.

The purpose of being applied in an educational setting is born from the authors' experience in having great difficulty identifying simple, notation-unbound tools for the explanation of the purposes and functionality of feature models. The existing tools are for the most part bound by particular notations or development environments, and often require download and installation to function, which while usually platform-independent, poses an extra step against the incentives of teaching new SPL notions to inexperienced and possibly unmotivated students.

In regards to its use as a repository, Atlas draws much inspiration from S.P.L.O.T. \cite{MENDONCA:2009} in attempting to make available a vast number of example feature models, previously validated and analysed, for purposes of serving as proofs of concept or reference specifications. The authors hope to make Atlas a centralizing reference for the education of multiple feature modeling notations through a primarily pedagogical public repository. Furthermore, the authors' experience in distributed research teams drives the need for an account-based private repository to be used in sharing models between contributors, enabling the editing and visualization of models by multiple individuals in geographically separate locations without the need for constant transfer of data, such as through e-mail attachments or cloud-based repositories like Google Drive or Dropbox.

The Atlas Project is part of the overarching Hestia Project, which proposes a pedagogical and research-oriented framework for the dissemination and development of SPL practices. Atlas is the first tool to be implemented in the author's plan of Olympus, a platform to support Hestia as both a Software Product Line (SPL) and Component-Based Software Engineering (CBSE) architectural framework.

The first installments of Atlas are to be developed by a team of undergraduate students from UNIPAMPA (Brazil) who are participating in a research internship programme, and will therefore serve as an experiment for the development of the Hestia framework itself, which is directed towards distributed, volatile and inexperienced teams. The project experiences will be documented so as to permit later analysis and to serve as input to the main Hestia Project.

We begin this document by describing the context within which this project was conceived and the objectives it hopes to achieve (Chapter \ref{chap:background}). We then proceed to elicit the requirements of the system, based on the context and objectives already raised and with the notion in mind of Atlas serving as a subsystem of the Olympus Project (Chapter \ref{chap:requirements}).

Following this, we present the summary of our studies on Feature Model notations, presenting the notations selected for representation on Atlas and the algorithms to enable conversion between notations based on their commonalities (Chapter \ref{chap:notations}). We then establish the architecture for Atlas, going through the selected technology used for the implementation of the project, the architectural styles and patterns used, the structural and behavioural descriptions of the system, and finally the rationale behind the decisions made \cite{PERRY:1992} (Chapter \ref{chap:architecture}).

Finally, we present the implemented tool, with a focus on its descriptive architecture and functionality, which is analysed based on the testing logs derived from the project's test cases (Chapter \ref{chap:atlas}). At the end of the document, the lessons learned from this project are listed alongside the conclusions drawn from the experience and which will be passed forward into the Hestia Project (Chapter \ref{chap:conclusion}).