\chapter{Background}
\label{chap:background}

The Atlas project, within the context of the Hestia macro-project, was born from the experiences of our research group over a few years studying Software Product Lines (SPL). It first appeared as the desire from the group to create a single tool that would enable it to more smoothly manage what it perceived as stages of design and development of an SPL, in order to serve as an auxiliary environment in the development of other research projects \cite{RODRIGUES:2014} \cite{RODRIGUES:2010}.

The ideas behind this tool quickly grew to a proportion that far exceeded the initial design decisions of the research group, first through a proposal of a software architecture that exceeded the original tool's capabilities \cite{LASER:2015} and then through the realizations that a framework could be constructed to aid in the several stages of SPL design and development which drew from our research and educational experience, in order not only to aid in particular tasks but to extend the adoption of SPL practices in general \cite{LASER:2015b} \cite{LASER:2016}.

Atlas is conceived as a solution for the particular research scenario lived by this group, and which the authors believe is a common scenario for research groups around the world. This scenario is characterized by contributors distributed in different geographical locations, with different schedules and working hours, with development teams that are inexperienced and volatile (in the sense that contributor turnaround is high) and limited financial resources. The requirements and design decisions of Atlas are representative of this particular scenario, and are aimed at solving the problems encountered by our group over the years. It is our firm belief that we are not alone in these constraints, and that the adoption of SPL practices would greatly benefit from a solution thus directed.

It is important to note that Atlas is meant as only one subsystem within the Olympus integrated environment, and is therefore designed with inter-system interactions in mind. While the authors believe that the contributions of this tool have merit in their own right, the requirements and design decisions taken throughout this document illustrate the objective of later integrating Atlas to other tools within the larger scope of the Hestia macro-project.

The following section lists the overall objectives of both the Atlas project and the Atlas tool, in order to serve as a foundation for the decisions taken throughout this project.

%----------------------------------------------

\section{Objectives}

In accordance to the context given above, the objectives of the Atlas tool can be listed as follows:

\begin{itemize}
    \item To provide a feature modeling environment that will be ideal for educational settings in which students may have little to no experience with SPL practices and may be initially unmotivated regarding the subject;
    \item To provide a feature model repository that will serve as a database of pre-validated example feature models in various notations, presenting various examples of the commonalities and peculiarities of each supported notation;
    \item To provide a feature modeling environment that is accessible to inexperienced users, possibly from fields other than software engineering or computer science in general, therefore furthering the adoption of SPL practices by a wider group of practitioners.
\end{itemize}

The following list presents the objectives of the Atlas project, within the context of the Hestia macro-project:

\begin{itemize}
    \item To examine the needs of geographically distributed research groups in regards to sharing and labeling of data related to feature models, in order to further explore solutions for geographically distributed SPL research in particular, and computer science research in general;
    \item To evaluate the perceived architectural practices in both design and implementation that will serve as the basis for the definition of the Hestia architectural style and its related patterns, in order to refine the ongoing research in this subject;
    \item To research and explore the fields of Human-Computer Interaction and Graphical User Interfaces, invaluable to the success of any Integrated Development Environment (IDE) or Computer-aided Software Engineering (CASE) tool;
    \item To establish the basis of the Olympus integrated environment, experiencing the architectural peculiarities of distributed and GUI-intensive web systems.
\end{itemize}